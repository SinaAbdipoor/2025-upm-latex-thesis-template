% info.tex
% Last reviewed: 9 Oct 2024 by Sina Abdipoor
% TODO: Fill out your information in the second brackets {}. The data will automatically be reflected in all necessary parts of the thesis.

% Title of your thesis
\newcommand{\infothesistitle}{Title of Thesis}

% Your name
\newcommand{\infostudentname}{Name of Student}

% Your pronoun
\newcommand{\infostudentpronoun}{his}

% Your degree (eg. Doctor of Philosophy)
\newcommand{\infodegreename}{Name of Degree}

% Name of your faculty
\newcommand{\infofacultyname}{Name of Faculty}

% Year of your viva
\newcommand{\infovivayear}{Year}

% Month of your viva (eg. September)
\newcommand{\infovivamonth}{Month}

% Day of your viva
\newcommand{\infovivaday}{Day}

% Name of your supervisor
\newcommand{\infosupervisorname}{Name of Chairman of Supervisory Committee}

% Degree of your supervisor (leave empty if not applicable)
\newcommand{\infosupervisordegree}{, PhD}

% Title of your supervisor
\newcommand{\infosupervisortitle}{Title (e.g., Professor/Associate Professor/Ir; if applicable)}

% Faculty of your supervisor
\newcommand{\infosupervisorfaculty}{Name of Faculty}

% Name of your committee 1
\newcommand{\infocommitteeonename}{Name of Member 1}

% Degree of your committee 1 (leave empty if not applicable)
\newcommand{\infocommitteeonedegree}{, PhD}

% Title of your committee 1
\newcommand{\infocommitteeonetitle}{Title (e.g., Professor/Associate Professor/Ir; if applicable)}

% Faculty of your committee 1
\newcommand{\infocommitteeonefaculty}{Name of Faculty}

% Name of your committee 2
\newcommand{\infocommitteetwoname}{Name of Member 2}

% Degree of your committee 2 (leave empty if not applicable)
\newcommand{\infocommitteetwodegree}{, PhD}

% Title of your committee 2
\newcommand{\infocommitteetwotitle}{Title (e.g., Professor/Associate Professor/Ir; if applicable)}

% Faculty of your committee 2
\newcommand{\infocommitteetwofaculty}{Name of Department and/or Faculty}

% University of your committee 2
\newcommand{\infocommitteetwouniversity}{Name of Organisation (University / Institute)}

% Name of your committee 3 (leave the second bracket empty if you only have 3 members)
\newcommand{\infocommitteethreename}{Name of Member 3}

% Degree of your committee 3 (delete if not applicable) (leave the second bracket empty if you only have 3 members)
\newcommand{\infocommitteethreedegree}{, PhD}

% Title of your committee 3 (leave the second bracket empty if you only have 3 members)
\newcommand{\infocommitteethreetitle}{Title (e.g., Professor/Associate Professor/Ir; if applicable)}

% Faculty of your committee 3 (leave the second bracket empty if you only have 3 members)
\newcommand{\infocommitteethreefaculty}{Name of Department and/or Faculty}

% University of your committee 3 (leave the second bracket empty if you only have 3 members)
\newcommand{\infocommitteethreeuniversity}{Name of Organisation (University / Institute)}

% Abstract of your thesis
\newcommand{\infoabstractenglish}{The abstract is a digest of the entire thesis and should be given the same consideration as the main text. It does not normally include any reference to the literature. Abbreviations or acronyms must be preceded by the full term at the first use.

An abstract should be between 300-500 words. It includes a brief statement of the problem, a concise description of the research method and design, a summary of major findings, including their significance or lack of it, and conclusions.}

% Keywords of your thesis
\newcommand{\infokeywords}{Not more than 5 keywords in alphabetical order must be provided to describe the content of the thesis}

% SDG
\newcommand{\infosdg}{GOAL 4: Quality Education (as an example) - Not more than 3 goal}

% Title of your thesis in Malay
\newcommand{\infothesistitlemalay}{Tajuk Tesis}

% Name of your degree in Malay
\newcommand{\infodegreenamemalay}{Nama Ijazah}

% Month and year of your viva in Malay
\newcommand{\infovivadatemalay}{Bulan Tahun}

% Name of your faculty in Malay
\newcommand{\infofacultynamemalay}{Nama Fakulti}

% Abstract of your thesis in Malay
\newcommand{\infoabstractmalay}{Abstrak merupakan ringkasan keseluruhan tesis dan wajib diberi perhatian rapi sepertimana bahagian tesis yang lain. Abstrak tidak mengandungi bahan rujukan. Nama singkatan atau akronim mesti didahului dengan terminology penuh pada penggunaan kali pertama.

Abstrak harus diolah antara 300-500 perkataan. Abstrak merangkumi pernyataan permasalahan, penerangan rigkas dan tepat tentang reka bentuk dan pengkaedahan penyelidikan, rumusan penemuan utama dan kesimpulan.}

% Keywords in Malay
\newcommand{\infokeywordsmalay}{Tidak lebih daripada 5 kata kunci dalam susunan abjad perlu disediakan untuk menerangkan kandungan tesis}

% SDG in Malay
\newcommand{\infosdgmalay}{MATLAMAT 4: Pendidikan Berkualiti (sebagai contoh) – Tidak lebih dari 3 matlamat}

% Your acknowledgments
\newcommand{\infoacks}{Acknowledgements are written expressions of appreciation for guidance and assistance received from individuals and institutions.}

% Name of the chairperson of the examination committee
\newcommand{\infochairpersonname}{Name of Chairperson}

% Degree of the chairperson of the examination committee (leave empty if not applicable)
\newcommand{\infochairpersondegree}{, PhD}

% Title of the chairperson of the examination committee
\newcommand{\infochairpersontitle}{Title (e.g., Professor/Associate Professor/Ir; omit if irrelevant)}

% Faculty of the chairperson of the examination committee
\newcommand{\infochairpersonfaculty}{Name of Faculty}

% Name of the examiner 1 of the examination committee
\newcommand{\infoexamineronename}{Name of Examiner 1}

% Degree of the examiner 1 of the examination committee (leave empty if not applicable)
\newcommand{\infoexamineronedegree}{, PhD}

% Title of the examiner 1 of the examination committee
\newcommand{\infoexamineronetitle}{Title (e.g., Professor/Associate Professor/Ir; omit if irrelevant)}

% Faculty of the examiner 1 of the examination committee
\newcommand{\infoexamineronefaculty}{Name of Faculty}

% Name of the examiner 2 of the examination committee
\newcommand{\infoexaminertwoname}{Name of Examiner 2}

% Degree of the examiner 2 of the examination committee (leave empty if not applicable)
\newcommand{\infoexaminertwodegree}{, PhD}

% Title of the examiner 2 of the examination committee
\newcommand{\infoexaminertwotitle}{Title (e.g., Professor/Associate Professor/Ir; omit if irrelevant)}

% Faculty of the examiner 2 of the examination committee
\newcommand{\infoexaminertwofaculty}{Name of Faculty}

% Name of the external examiner of the examination committee
\newcommand{\infoexternalexaminername}{Name of External Examiner}

% Degree of the external examiner of the examination committee (leave empty if not applicable)
\newcommand{\infoexternalexaminerdegree}{, PhD}

% Title of the external examiner of the examination committee
\newcommand{\infoexternalexaminertitle}{Title (e.g., Professor/Associate Professor/Ir; omit if irrelevant)}

% Department and faculty of the external examiner of the examination committee
\newcommand{\infoexternalexaminerdepartment}{Name of Department and/or Faculty}

% University of the external examiner of the examination committee
\newcommand{\infoexternalexamineruniversity}{Name of Organisation (University/Institute)}

% Country of the external examiner of the examination committee
\newcommand{\infoexternalexaminercountry}{Country}

% Name of current Dean
\newcommand{\infodeanname}{Current Dean Name}

% Degree of the current Dean (leave empty if not applicable)
\newcommand{\infodeandegree}{, PhD}

% Name of current Deputy Dean
\newcommand{\infodeputydeanname}{Current Deputy Dean Name}

% Degree of the current Deputy Dean (leave empty if not applicable)
\newcommand{\infodeputydeandegree}{, PhD}

% Your Biodata
\newcommand{\infostudentbiodata}{This section is compulsory. It contains the student’s biographical information, such as name, educational background, the degree that is being sought, professional work experience (if any), and any other similar matters that may interest the reader. The vita should be in essay form, rather than a mere résumé.}